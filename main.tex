\documentclass[xcolor=dvipsnames]{beamer}
\useinnertheme{rectangles}
\useoutertheme{infolines}

\usepackage[portuguese]{babel}
\usepackage[utf8]{inputenc}
\usepackage{url}

\title{Hardware e Sistemas Operacionais}
\subtitle{15/06/2021, 11h - 12h}
\author{Josiney de Souza (josiney.souza)}
\institute{Instituto Federal Catarinense campus Brusque}
\date{\today}

%%% Apagar barra de navegacao:
% https://stackoverflow.com/questions/3210205/how-to-get-rid-of-navigation-bars-in-beamer
\beamertemplatenavigationsymbolsempty
% https://stackoverflow.com/questions/1435837/how-to-remove-footers-of-latex-beamer-templates
\beamertemplatefootempty

%%% Cores das fontes:
% https://en.wikibooks.org/wiki/LaTeX/Colors
\definecolor{ifvermelho}{RGB}{200,12,15}
\definecolor{ifverde}{HTML}{2F9E41}

% https://tex.stackexchange.com/questions/133820/beamer-how-to-change-color-of-infolines-and-frame-title
\setbeamercolor{title}{fg=Red}
\setbeamercolor{frametitle}{fg=Red}
\setbeamercolor{section in head/foot}{fg=White,bg=ifvermelho}
\setbeamercolor{subsection in head/foot}{fg=ifverde}

\begin{document}
\begin{frame}[plain]{}
    \maketitle
\end{frame}

{\small
\begin{frame}[plain]{Sumário}
    \tableofcontents
\end{frame}
}

\section{Seção 1}
\subsection{Texto}
\begin{frame}{Slide 1}
	Texto do slide 1
\end{frame}

\subsection{Lista}
\begin{frame}{Slide 2}
	\begin{itemize}
		\item Item 1
		\item Item 2
		\item Item 3
		\item Item 4
		\item Item 5
	\end{itemize}
\end{frame}

\section{Seção 2}
\subsection{Definições}
\begin{frame}{Slide 3}
	\begin{block}{Título do bloco}
		Definição no bloco
	\end{block}

	\begin{exampleblock}{Título do bloco}
		Definição no bloco
	\end{exampleblock}

	\begin{alertblock}{Título do bloco}
		Definição no bloco
	\end{alertblock}
\end{frame}

\subsection{Imagem/Figura}
\begin{frame}{Slide 4}
	Figura aqui\footnote{teste}
\end{frame}

\section{Seção 3}
\subsection{Tabela}
\begin{frame}{Slide 5}
	\begin{tabular}{|c|c|c|}
		\hline
		Num & Nome & login \\
		\hline
		\hline
		1 & Josiney & jos04 \\
		\hline
		2 & Leonardo & leg04 \\
		\hline
		3 & Rubens & rums04 \\
		\hline
	\end{tabular}
\end{frame}
	
\end{document}