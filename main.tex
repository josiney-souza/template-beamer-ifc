\documentclass[xcolor=dvipsnames,table]{beamer}
\useinnertheme{rounded}
\useoutertheme{infolines}

\usepackage[portuguese]{babel}
\usepackage[utf8]{inputenc}
\usepackage{url}

\title{Hardware e Sistemas Operacionais}
\subtitle{15/06/2021, 11h - 12h}
\author{Josiney de Souza (josiney.souza)}
\institute{Instituto Federal Catarinense campus Brusque}
\date{\today}

%%% Apagar barra de navegacao:
% https://stackoverflow.com/questions/3210205/how-to-get-rid-of-navigation-bars-in-beamer
% https://stackoverflow.com/questions/1435837/how-to-remove-footers-of-latex-beamer-templates
\beamertemplatenavigationsymbolsempty
%%% Remover o titulo, autor e data do rodape:
% https://tex.stackexchange.com/questions/113443/remove-author-and-institution-in-footline
\beamertemplatefootempty

%%% Cores das fontes:
% https://en.wikibooks.org/wiki/LaTeX/Colors
\definecolor{ifvermelho}{RGB}{200,12,15}
\definecolor{ifverde}{HTML}{2F9E41}

% https://tex.stackexchange.com/questions/133820/beamer-how-to-change-color-of-infolines-and-frame-title
\setbeamercolor{title}{fg=Red}
\setbeamercolor{frametitle}{fg=Red}
\setbeamercolor{section in head/foot}{fg=White,bg=ifvermelho}
\setbeamercolor{subsection in head/foot}{fg=ifverde}

%%% Mostrar o sumario novamente antes do inicio da proxima secao
% https://pt.overleaf.com/learn/latex/beamer
\AtBeginSection[]
{\small
  \begin{frame}[plain]
    \frametitle{Sumário}
    \tableofcontents[currentsection]
  \end{frame}
}

%%% Plano de fundo usado no modelo ODT/DOC/DOCX da Cecom
\usebackgroundtemplate%
{%
    \includegraphics[width=\paperwidth,height=\paperheight]{fundo.jpg}%
}



%%%%%%%%%%%%%%%%%%%%%%%%%%%%%%%%%%%%%%%%%%%%%%%%%%%
%%%%%%%%%%%%%%% Início do documento %%%%%%%%%%%%%%%
%%%%%%%%%%%%%%%%%%%%%%%%%%%%%%%%%%%%%%%%%%%%%%%%%%%
\begin{document}

%%%%% Primeira página do modelo abaixo - PODE SER REMOVIDA, SE DESEJAR %%%%%
%%% Centralizando o titulo da primeira pagina
% https://stackoverflow.com/questions/2365539/centering-titles-when-using-the-beamer-class-in-latex
%%% Removendo as formatacoes do slide
% https://tex.stackexchange.com/questions/281334/left-aligned-title-page-in-beamer-boadilla-usetheme
{
\usebackgroundtemplate%
{%
    \includegraphics[width=\paperwidth,height=\paperheight]{primeira-pag-menor.jpg}%
}
\begin{frame}[plain]{\centerline{\inserttitle}}
\end{frame}
}
%%%%% Primeira página do modelo acima - PODE SER REMOVIDA, SE DESEJAR %%%%%

\begin{frame}[plain]{}
    \maketitle
\end{frame}

{\small
\begin{frame}[plain]{Sumário}
    \tableofcontents
\end{frame}
}

\section{Seção 1}
\subsection{Texto}
\begin{frame}{Slide 1}
	Texto do slide 1
\end{frame}

\subsection{Lista}
\begin{frame}{Slide 2}
	\begin{itemize}
		\item Item 1
		\item Item 2
		\item Item 3
		\item Item 4
		\item Item 5
	\end{itemize}
\end{frame}

\section{Seção 2}
\subsection{Definições}
\begin{frame}{Slide 3}
	\begin{block}{Título do bloco}
		Definição no bloco
	\end{block}

	\begin{exampleblock}{Título do bloco}
		Definição no bloco
	\end{exampleblock}

	\begin{alertblock}{Título do bloco}
		Definição no bloco
	\end{alertblock}
\end{frame}

\subsection{Imagem/Figura}
\begin{frame}{Slide 4}
	Figura aqui\footnote{Uma imagem da Internet}

	\begin{center}
		\includegraphics[scale=0.25]{bios2.jpg}
	\end{center}
\end{frame}

\section{Seção 3}
\subsection{Tabela}
\begin{frame}{Slide 5}
	\begin{tabular}{|c|c|c|}
		\hline
		Num & Nome & login \\
		\hline
		\hline
		1 & Josiney & jos04 \\
		\hline
		2 & Leonardo & leg04 \\
		\hline
		3 & Rubens & rums04 \\
		\hline
	\end{tabular}
\end{frame}

%%%%% Última página do modelo abaixo - PODE SER REMOVIDA, SE DESEJAR %%%%%
{
\usebackgroundtemplate%
{%
    \includegraphics[width=\paperwidth,height=\paperheight]{ultima-pag-menor.jpg}%
}
\begin{frame}[plain]{}
\end{frame}
}
%%%%% Última página do modelo acima - PODE SER REMOVIDA, SE DESEJAR %%%%%
	
\end{document}